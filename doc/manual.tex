% generated by GAPDoc2LaTeX from XML source (Frank Luebeck)
\documentclass[a4paper,11pt]{report}

\usepackage{a4wide}
\sloppy
\pagestyle{myheadings}
\usepackage{amssymb}
\usepackage[latin1]{inputenc}
\usepackage{makeidx}
\makeindex
\usepackage{color}
\definecolor{FireBrick}{rgb}{0.5812,0.0074,0.0083}
\definecolor{RoyalBlue}{rgb}{0.0236,0.0894,0.6179}
\definecolor{RoyalGreen}{rgb}{0.0236,0.6179,0.0894}
\definecolor{RoyalRed}{rgb}{0.6179,0.0236,0.0894}
\definecolor{LightBlue}{rgb}{0.8544,0.9511,1.0000}
\definecolor{Black}{rgb}{0.0,0.0,0.0}

\definecolor{linkColor}{rgb}{0.0,0.0,0.554}
\definecolor{citeColor}{rgb}{0.0,0.0,0.554}
\definecolor{fileColor}{rgb}{0.0,0.0,0.554}
\definecolor{urlColor}{rgb}{0.0,0.0,0.554}
\definecolor{promptColor}{rgb}{0.0,0.0,0.589}
\definecolor{brkpromptColor}{rgb}{0.589,0.0,0.0}
\definecolor{gapinputColor}{rgb}{0.589,0.0,0.0}
\definecolor{gapoutputColor}{rgb}{0.0,0.0,0.0}

%%  for a long time these were red and blue by default,
%%  now black, but keep variables to overwrite
\definecolor{FuncColor}{rgb}{0.0,0.0,0.0}
%% strange name because of pdflatex bug:
\definecolor{Chapter }{rgb}{0.0,0.0,0.0}
\definecolor{DarkOlive}{rgb}{0.1047,0.2412,0.0064}


\usepackage{fancyvrb}

\usepackage{mathptmx,helvet}
\usepackage[T1]{fontenc}
\usepackage{textcomp}


\usepackage[
            pdftex=true,
            bookmarks=true,        
            a4paper=true,
            pdftitle={Written with GAPDoc},
            pdfcreator={LaTeX with hyperref package / GAPDoc},
            colorlinks=true,
            backref=page,
            breaklinks=true,
            linkcolor=linkColor,
            citecolor=citeColor,
            filecolor=fileColor,
            urlcolor=urlColor,
            pdfpagemode={UseNone}, 
           ]{hyperref}

\newcommand{\maintitlesize}{\fontsize{50}{55}\selectfont}

% write page numbers to a .pnr log file for online help
\newwrite\pagenrlog
\immediate\openout\pagenrlog =\jobname.pnr
\immediate\write\pagenrlog{PAGENRS := [}
\newcommand{\logpage}[1]{\protect\write\pagenrlog{#1, \thepage,}}
%% were never documented, give conflicts with some additional packages

\newcommand{\GAP}{\textsf{GAP}}

%% nicer description environments, allows long labels
\usepackage{enumitem}
\setdescription{style=nextline}

%% depth of toc
\setcounter{tocdepth}{1}



\usepackage[pdftex]{graphicx}

%% command for ColorPrompt style examples
\newcommand{\gapprompt}[1]{\color{promptColor}{\bfseries #1}}
\newcommand{\gapbrkprompt}[1]{\color{brkpromptColor}{\bfseries #1}}
\newcommand{\gapinput}[1]{\color{gapinputColor}{#1}}


\begin{document}

\logpage{[ 0, 0, 0 ]}
\begin{titlepage}
\mbox{}\vfill

\begin{center}{\maintitlesize \textbf{\textsf{FSR}\mbox{}}}\\
\vfill

\hypersetup{pdftitle=\textsf{FSR}}
\markright{\scriptsize \mbox{}\hfill \textsf{FSR} \hfill\mbox{}}
{\Huge \textbf{...\mbox{}}}\\
\vfill

{\Huge Version 1.0.3\mbox{}}\\[1cm]
{16 January 2017\mbox{}}\\[1cm]
\mbox{}\\[2cm]
{\Large \textbf{Nusa Zidaric   \mbox{}}}\\
\hypersetup{pdfauthor=Nusa Zidaric   }
\end{center}\vfill

\mbox{}\\
{\mbox{}\\
\small \noindent \textbf{Nusa Zidaric   }  Email: \href{mailto://email} {\texttt{email}}\\
  Homepage: \href{http://} {\texttt{http://}}}\\
\end{titlepage}

\newpage\setcounter{page}{2}
{\small 
\section*{Abstract}
\logpage{[ 0, 0, 1 ]}
 \index{FSR package@\textsf{FSR} package} The \textsf{GAP} package \textsf{FSR} ... \mbox{}}\\[1cm]
{\small 
\section*{Copyright}
\logpage{[ 0, 0, 2 ]}
 {\copyright} 2017-2017 by Nusa Zidaric

 \textsf{FSR} is free software; you can redistribute it and/or modify it under the terms of
the GNU General Public License as published by the Free Software Foundation;
either version 2 of the License, or (at your option) any later version. For
details, see the FSF's own site \href{http://www.gnu.org/licenses/gpl.html} {\texttt{http://www.gnu.org/licenses/gpl.html}}. 

 If you obtained \textsf{FSR}, we would be grateful for a short notification sent to one of the authors. 

 If you publish a result which was partially obtained with the usage of \textsf{FSR}, please cite it in the following form: 

 N. Zidaric. ... \mbox{}}\\[1cm]
{\small 
\section*{Acknowledgements}
\logpage{[ 0, 0, 3 ]}
 ... \mbox{}}\\[1cm]
\newpage

\def\contentsname{Contents\logpage{[ 0, 0, 4 ]}}

\tableofcontents
\newpage

 
\chapter{\textcolor{Chapter }{Preface}}\label{Preface}
\logpage{[ 1, 0, 0 ]}
\hyperdef{L}{X874E1D45845007FE}{}
{
  The \textsf{GAP} package \textsf{FSR} implements ... }

 
\chapter{\textcolor{Chapter }{Output formatting functions}}\label{Outputs}
\logpage{[ 2, 0, 0 ]}
\hyperdef{L}{X7EB095598359B4B3}{}
{
  This is some intro... 
\section{\textcolor{Chapter }{Formatting ...}}\label{Formatting}
\logpage{[ 2, 1, 0 ]}
\hyperdef{L}{X866C354883562E3D}{}
{
  Some intro ... 

\subsection{\textcolor{Chapter }{ IntFFExt}}
\logpage{[ 2, 1, 1 ]}\nobreak
\hyperdef{L}{X7C848712781083D5}{}
{\noindent\textcolor{FuncColor}{$\triangleright$\ \ \texttt{ IntFFExt({\mdseries\slshape [B, ]ffe})\index{ IntFFExt@\texttt{ IntFFExt}}
\label{ IntFFExt}
}\hfill{\scriptsize (function)}}\\
\noindent\textcolor{FuncColor}{$\triangleright$\ \ \texttt{ IntVecFFExt({\mdseries\slshape [B, ]vec})\index{ IntVecFFExt@\texttt{ IntVecFFExt}}
\label{ IntVecFFExt}
}\hfill{\scriptsize (function)}}\\
\noindent\textcolor{FuncColor}{$\triangleright$\ \ \texttt{ IntMatFFExt({\mdseries\slshape [B, ]M})\index{ IntMatFFExt@\texttt{ IntMatFFExt}}
\label{ IntMatFFExt}
}\hfill{\scriptsize (function)}}\\


 IntFFExt takes the \mbox{\texttt{\mdseries\slshape ffe}} and writes it as an integer of the prime field if \mbox{\texttt{\mdseries\slshape ffe}} is an element of the prime field (same as Int(ffe)), or writes it as a vector
of integers from the prime subfield if \mbox{\texttt{\mdseries\slshape ffe}} is an element of an extension field, using the given basis \mbox{\texttt{\mdseries\slshape B}} or canonical basis representation of \mbox{\texttt{\mdseries\slshape ffe}}. 

 IntVecFFExt takes the vector \mbox{\texttt{\mdseries\slshape vec}} of FFEs and writes it as a vector of integers from the prime field if all
components of \mbox{\texttt{\mdseries\slshape vec}} belong to a prime field, or as a vector of vectors of integers from the prime
subfield, if the components belong to an extension field, using the given
basis \mbox{\texttt{\mdseries\slshape B}} or canonical basis representation of \mbox{\texttt{\mdseries\slshape ffe}}. (note: all components are treated as elements of the largest field). 

 IntMATFFExt takes a matrix \mbox{\texttt{\mdseries\slshape M}} and returns a matrix of vectors of integers from the prime field if all
components of \mbox{\texttt{\mdseries\slshape M}} belong to a prime field, or a vector of row vectors, whose elements are
vectors of integers from the prime subfield, if the components belong to an
extension field, using the given basis \mbox{\texttt{\mdseries\slshape B}} or canonical basis representation of \mbox{\texttt{\mdseries\slshape ffe}}. 

 NOTE: the non-basis versions return a representation in the SMALLEST field
that contains the element. for representation in a specific field, use the
basis version with desired basis }

 

\subsection{\textcolor{Chapter }{VecToString}}
\logpage{[ 2, 1, 2 ]}\nobreak
\hyperdef{L}{X78387B8B7940F96C}{}
{\noindent\textcolor{FuncColor}{$\triangleright$\ \ \texttt{VecToString({\mdseries\slshape [B, ]vec})\index{VecToString@\texttt{VecToString}}
\label{VecToString}
}\hfill{\scriptsize (function)}}\\


 writes a (FFE) verctor as string or list of strings using the given basis \mbox{\texttt{\mdseries\slshape B}} or canonical basis representation of \mbox{\texttt{\mdseries\slshape ffe}}. The list of strings is more practically useful: we wish to have the
components as srings, which is what we want to use }

 }

 }

 \def\bibname{References\logpage{[ "Bib", 0, 0 ]}
\hyperdef{L}{X7A6F98FD85F02BFE}{}
}

\bibliographystyle{alpha}
\bibliography{manual}

\addcontentsline{toc}{chapter}{References}

\def\indexname{Index\logpage{[ "Ind", 0, 0 ]}
\hyperdef{L}{X83A0356F839C696F}{}
}

\cleardoublepage
\phantomsection
\addcontentsline{toc}{chapter}{Index}


\printindex

\newpage
\immediate\write\pagenrlog{["End"], \arabic{page}];}
\immediate\closeout\pagenrlog
\end{document}
